\documentclass[10pt]{article}
\usepackage{amsmath}
\usepackage{geometry}
\usepackage{fancybox}
\usepackage{amsfonts}
\usepackage{tikz}
\usepackage{listings}
 \geometry{
 a4paper,
 total={170mm,257mm},
 left=20mm,
 top=10mm,
 bottom=15mm
 }
 
\usepackage{color}
 
\definecolor{codegreen}{rgb}{0,0.6,0}
\definecolor{codegray}{rgb}{0.5,0.5,0.5}
\definecolor{codepurple}{rgb}{0.58,0,0.82}
\definecolor{backcolour}{rgb}{0.95,0.95,0.92}

\linespread{1.3}

\title{CSC263H1 Assignment 4}
\author{Jiatao Xiang, Xu Wang, Huakun Shen}
\date{February 28th, 2019}

\begin{document}
\maketitle
\section*{Question 1}
\begin{enumerate}
\item[a.] There is a probability of $\frac{k}{n}$ for the algorithm to return \textbf{TRUE} in the first iteration.\\
There are $n$ integers in total, where $k$ of them are $x$. Picking one from them yields $\frac{k}{n}$ of probability to get an $x$.

\item[b.] Geometric Distribution. The probability for the algorithm to return \textbf{TRUE} in $r$ iterations is,
\begin{align*}
P(x\leq r)&=(1-\frac{k}{n})^0\cdot \frac{k}{n}+(1-\frac{k}{n})^1\cdot \frac{k}{n}+...+(1-\frac{k}{n})^{r-1}\cdot \frac{k}{n}\\
&=\Sigma^{r-1}_{i=0}((1-\frac{k}{n})^i\cdot \frac{k}{n})
\end{align*}
Where $x$ is the number iterations where the first success occurs.
\item[c.] When the algorithm is modified with an infinite loop, the expected number of loop iterations is $\frac{n}{k}-1$
\begin{align*}
E(x)&=\frac{1-\frac{k}{n}}{\frac{k}{n}}\\
&=(1-\frac{k}{n})\cdot\frac{n}{k}\\
&=\frac{n}{k}-1
\end{align*}
\end{enumerate}

\section*{Question 2}
\begin{enumerate}
\item
\item
\end{enumerate}


\end{document}